%%----------------------------------------------------------------------
%%----------------------------------------------------------------------
%\begin{storyquotepage}  
%  Even the smallest of daggers can cause the\\gravest of wounds if they
%  come from behind.
%
%\bigskip
%- Attribution lost,\\Chronicles of Foundation,\\restricted historical draft
%\end{storyquotepage}

\makeatletter\@openrightfalse
\chapter{Army Selection}
\@openrighttrue\makeatother

Each player must prepare two army lists:

\begin{itemize}
\item \textbf{Campaign Force.} For your individual battles, the
  majority of the campaign.\hfill\textit{Up to 2000 points.}

\item \textbf{Strike Force.} For your team games, of which there will
  be at least one.\hfill\textit{Up to 1000 points.}
\end{itemize}

Both army lists must be battle forged.  No model with more than~9 hull
points or wounds is permitted.  Superheavy vehicles and gargantuan
creatures are otherwise allowed.

The strike force list need not be a subset of the campaign force list,
but cannot include any factions not utilized in the campaign force
list.  As detailed below, in team games you will have your own
warlord, and your units will count as allies of convenience to those
of your partner(s), regardless of factions.

No other requirements or constraints are placed on detachments,
formations, or force organization. An optional Quick Reaction Force
detachment is made available for this event, described below.

\section{Sources}

Forgeworld units and armies eligible for standard \textit{Warhammer
  40,000}, i.e., not \textit{Apocalypse}, are permitted. Units and
armies from Forgeworld's Horus Heresy \textit{Age of Darkness} books
are also permitted.

All up-to-date, official \textit{Warhammer 40,000} army sources are
permitted that are available in current publication. This does include
White Dwarf entries, which are available via back issues, and current
campaign books. It does not include limited edition dataslates and
formations, e.g., those included in mega-bundles. Contact the
tournament organizer(s) beforehand about any questions. Remember that
you must have all sources on hand, electronically or digitally.

For any codex or supplement re-released within two weeks preceding the
event, you may choose whether to use the old or new edition. You may
not use both editions of a single source within the event.

\section{Models}

Models must be WYSIWYG, but identifiable and thoughtful conversions
and proxies are welcome. Indistinguishable or confusing proxies are
not acceptable.  Contact the organizer(s) beforehand for any
questions.

In addition, models need not be painted, but is \textit{very strongly}
recommended in order to not impair the experiences of all other
participants.  A painting component will be applied to personal scores
to reward finished armies, following the standardized NOVA metrics.


\clearpage
\section{Quick Reaction Force}

Players may optionally employ a Quick Reaction Force detachment in
their army lists, defined as follows.

\subsection{Force Organization}

An army may only contain a single Quick Reaction Force detachment.
All units in the detachment must have the same faction, or no faction.
The detachment is comprised of the following battlefield roles.

\begin{center}
\begin{tabular}{C{0.75in}C{0.75in}C{0.75in}C{0.75in}C{0.75in}C{0.75in}}
\rowcolor{LineColor}\textbf{\color{white} HQ}	& \textbf{\color{white} Troops}	& \textbf{\color{white} Elites}	& \textbf{\color{white} Fast Attack}	& \textbf{\color{white} Heavy Support}	& \textbf{\color{white} Lords of War}\\
1--2	& 2--6	& 1--4	& $\star$	& $\star$	& 0--1\\
\end{tabular}
\end{center}

A Quick Reaction Force detachment must include one Fast Attack and one
Heavy Support choice. It may include up to three selections in one of
those roles, but must contain one and only one selection in the other
role. I.e., the detachment must adhere to one of the following
options:

\begin{itemize}
\item 1--3 Fast Attack and 1 Heavy Support
\item 1 Fast Attack and 1--3 Heavy Support
\end{itemize}

As usual, dedicated transports are not counted toward these quantity limitations.

\subsection{Command Benefits}

The following advantages are granted for utilizing this detachment:

\begin{itemize}
\item \textbf{Objective Secured.} All scoring units in this detachment
  except superheavy vehicles and gargantuan creatures gain Objective
  Secured. A unit with this special rule controls objectives even if
  an enemy scoring unit is within range of the objective marker,
  unless the enemy unit also has this special rule.

\item \textbf{QRF Commander.} If this detachment is your primary
  detachment, instead of rolling a random warlord trait you may choose
  a warlord trait from the tables in the main \emph{Warhammer 40,000}
  rulebook.  If you wish to use a different table of warlord traits,
  i.e., from a codex, then you must roll a random trait, but may
  reroll the result.

\end{itemize}
